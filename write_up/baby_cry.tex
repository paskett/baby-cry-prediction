\documentclass[10pt,letterpaper]{article}
\usepackage{fullpage}
\usepackage[top=1.75cm, bottom=2cm, left=2cm, right=2cm]{geometry}
\usepackage{amsmath,amsthm,amsfonts,amssymb,amscd}
\usepackage{lastpage}
\usepackage{enumerate}
\usepackage{fancyhdr}
\usepackage{mathrsfs}
\usepackage{xcolor}
\usepackage{graphicx}
\usepackage{listings}
\usepackage{hyperref}
\usepackage{tcolorbox}
\usepackage{bbm}

\hypersetup{%
colorlinks=true,
linkcolor=blue,
linkbordercolor={0 0 1}
}

\renewcommand\lstlistingname{Algorithm}
\renewcommand\lstlistlistingname{Algorithms}
\def\lstlistingautorefname{Alg.}

\lstdefinestyle{Python}{
  language        = Python,
  frame           = lines,
  basicstyle      = \footnotesize,
  keywordstyle    = \color{blue},
  stringstyle     = \color{green},
  commentstyle    = \color{red}\ttfamily
}

\setlength{\parindent}{0.0in}
\setlength{\parskip}{0.05in}

\pagestyle{fancyplain}
\headsep 1.5em
\title{Baby Cry Prediction}

\begin{document}
\maketitle
\begin{abstract}
   TODO - A one-paragraph abstract (tl;dr) is required. Summarize the main question and conclusions. Put this below the title but before the main text.
\end{abstract}

\section{Problem Statement and Motivation}
\textit{Give a clear statement of the problems or questions the project addresses, their context, and a compelling motivation for why they are worth studying. The problems or questions your project addresses should be original, meaningful, and reflect a deep understanding of the subject and its subtleties.}

\textit{Briefly review what is already known about the research questions and what techniques others have used to study these questions. Explain the scope of the project and how it fits into existing research.}


\section{Data}
\textit{Briefly discuss the sources, reliability, and suitability of your data for the problems you are addressing. The dataset should be large enough and rich enough to give reliable, meaningful, and nuanced answers to the questions and problems addressed in your project.}

\section{Methods}
\textit{Your project should involve a thoughtful and original use of several of the machine learning methods and ideas we have learned in class and at least one idea or method we have not covered in class. In this section you should describe and justify your selection of models, your choices of methods, and the ways and reasons you chose your hyperparameters or network architectures. You should also describe the additional method you chose to use, the mathematics of how it works, and why it might be well-suited to your dataset. Your discussion should demonstrate a clear understanding of the principles involved and of the strengths and weaknesses of the models and methods used.}

\textit{It should be noted that while the additional method you choose must not be one of the ones covered in class, it can be related to them. For example, even though we will cover both regular and convolution neural networks in class, you could still implement a different flavor of network (i.e. a recurrent neural network, an object detection network, etc.) for your additional algorithm. The purpose of this portion is to accustom you to reading and implementing current machine learning literature, since doing so is the only way to remain competitive in such a fast-moving field.}

\section{Results}
\textit{Clearly and succinctly describe your results.}

\section{Analysis}
\textit{Give a thorough analysis and a thoughtful discussion of the results and conclusions that can be drawn. Discuss the suitability and effectiveness of the different models and methods for the problems or questions treated.}

\section{Conclusion}
\textit{Briefly summarize what you have done and describe the final conclusions that you draw from your computations, results, and analysis.}


\end{document}
